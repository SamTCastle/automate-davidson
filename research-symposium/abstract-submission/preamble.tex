%&pdfLaTeX
% !TEX encoding = UTF-8 Unicode
\documentclass{article}
\usepackage{ifxetex}
\ifxetex
\usepackage{fontspec}
\setmainfont[Mapping=tex-text]{STIXGeneral}
\else
\usepackage[T1]{fontenc}
\usepackage[utf8]{inputenc}
\fi
\usepackage{textcomp}

\usepackage{amssymb}
\usepackage{fancyhdr}
\renewcommand{\headrulewidth}{0pt}
\renewcommand{\footrulewidth}{0pt}

% Choose page margins.
\usepackage{geometry}
\geometry{margin=.75in}
%\geometry{lmargin=1in}  % Change left margin.

\usepackage{fixltx2e}  % For \textsubscript{} command.

% For dots in table of contents.
\usepackage{tocloft}
\renewcommand{\cftsecleader}{\cftdotfill{\cftdotsep}}

% Package for capitalization of titles.
%\usepackage{titlecaps}
%\Addlcwords{the and a in as well of}

\usepackage{environ}

% Command to use titlecaps and bold.
%\newcommand{\boldtitlecap}[1]{\textbf{\titlecap{#1}}}

\usepackage{sectsty}  % Used to control font syling of section headings.
% Center headings and no hanging indent.
% Also make all upper case by reversing the MakeLowercase macro.
\allsectionsfont{\nohang\centering\scshape\MakeLowercase}

% Custom department section heading environ.
\NewEnviron{departmentheading}{%
	\pagebreak
	\addcontentsline{toc}{section}{\BODY}
	\section*{\BODY}
	\vspace{0.3cm}
	}

% Custom environment to format th title, author, and abstract for a single
% student's submission as a single unit.
% Primarily, this is to handle each title, author line, and abstract together
% and prevent page breaking within this unit.
% Use the samepage environment, a \parbox, a minipage, or the \needspace
% command to prevent page breaks within this unit.
\NewEnviron{titleauthorabstract}{%
	%\begin{minipage}{\textwidth}
	\filbreak
	\begin{samepage}
	\BODY
	\end{samepage}
	\vspace{0.7cm}
	}

% Custom title environ.
\NewEnviron{customtitle}{%
	%\titlecap{\textbf{\BODY}} % For capitalizing titles.
	\textbf{\BODY} 
	}

% Custom author environ.
\NewEnviron{customauthor}{%
	\textbf{\BODY}
	}

% Custom abstract environ.
\NewEnviron{customabstract}{%
	\vspace{0.3cm}
	\BODY
	}

%----------------------------------------------------------------------------------------
%	MAIN DOCUMENT
%----------------------------------------------------------------------------------------
\begin{document}
\raggedright{}

% Input the title page, from customtitle.tex.
% Note: customtitle.tex must be located in the working directory.
\newcommand*{\customdate}{06 May 2015}
% Define the date of the event.

%----------------------------------------------------------------------------------------
%	TITLE PAGE
%----------------------------------------------------------------------------------------

\begin{customtitle}
	\begin{center} % Center all text
		\vspace*{5\baselineskip} % White space at the top of the page

		\rule{\textwidth}{1.6pt}\vspace*{-\baselineskip}\vspace*{2pt} % Thick horizontal line
		\rule{\textwidth}{0.4pt}\\[\baselineskip] % Thin horizontal line

		% The main title.
		{\LARGE\bfseries Davidson College\\[0.4\baselineskip] Science \& Math Student Research Symposium}\\[0.2\baselineskip]

		\rule{\textwidth}{0.4pt}\vspace*{-\baselineskip}\vspace{3.2pt} % Thin horizontal line
		\rule{\textwidth}{1.6pt}\\[\baselineskip] % Thick horizontal line

		%\scshape % Small caps

		% Skip space and insert the date, defined above.
		\vspace{2\baselineskip}
		{\LARGE \customdate}

		\vspace*{2\baselineskip} % Whitespace between location/year and editors

		% Change name of table of contents and insert table of contents.
		\renewcommand{\contentsname}{\scshape Contents}
		{\flushleft\Large \tableofcontents}

		% Fill to bottom of page.
		\vfill

		% Insert Acknowledgment.
		\subsection*{Acknowledgment}
		% Flush left and remove bold.
		\begin{flushleft}
			\textnormal{The Vice President for Academic Affairs Office
			generously provided funds to support this important event.}
		\end{flushleft}

	\end{center}
\end{customtitle}

% Suppress page numbers on the intro page.
% Also make the following page begin with page number 1.
\thispagestyle{empty}
\clearpage
\setcounter{page}{1}